\documentclass[a4paper, english, 12pt]{article}
\usepackage[utf8]{inputenc}
\usepackage[T1]{fontenc}
\usepackage{babel, textcomp, color, amssymb, subfig, float}
\usepackage{amsfonts}
\usepackage{graphicx}
\usepackage{multicol}   
\usepackage{bm}
\usepackage{gensymb}
\usepackage{amsmath}
\usepackage{bbold}

\usepackage{physics}
\usepackage{tikz}
\usepackage{pgfplots}
\newcommand{\eps}{\epsilon}
\newcommand{\closed}[1]{\left( #1 \right)}
\newcommand{\bracket}[1]{\left[ #1 \right]}
\newcommand{\curlybig}[1]{\left\{ #1 \right\} }
\newcommand{\curly}[1]{\{ #1 \} }
\newcommand{\Q}{\mathbb{Q}}
\renewcommand{\P}{\mathbb{P}}
\newcommand{\Z}{\mathcal{Z}}

%\newcommand{\addPLOT}[4]{
%\addplot [domain=#1:#2,samples=200,color=#3,]{#4};}
%\newcommand{\addCOORDS}[1]{\addplot coordinates {#1};}
%\newcommand{\addDRAW}[1]{\draw #1;}
%\newcommand{\addNODE}[2]{ \node at (#1) {#2};}

%		\PLOTS{x}{y}{left}{
%			\ADDPLOT{x^2}{-2}{2}{blue}
%			\ADDCOORDS{(0,1)(1,1)(1,2)}
%		}


\definecolor{svar}{RGB}{0,0,0}
\definecolor{opgavetekst}{RGB}{109,109,109}
\definecolor{blygraa}{RGB}{44,52,59}

\hoffset = -60pt
\voffset = -95pt
\oddsidemargin = 0pt
\topmargin = 0pt
\textheight = 0.97 \paperheight
\textwidth = 0.97 \paperwidth

\begin{document}
\tiny
\begin{multicols*}{2}


\subsection*{\boxed{\text{Thermodynamic postulates}}}

\begin{enumerate}
    \setlength\itemsep{0.05em}
    \item \textbf{Equilibrium states:} There exists equilibrium states of a macroscopic system that are characterized by a small number of extensive variables. 
    \item \textbf{Entropy maximization:} The values assumed by the extensive parameters of an isolated composite system in the absence of an internal constraint are those that maximize the entropy over the set of all constrained macroscopic states. $(\Delta S \geq 0)$
    \item \textbf{Additivity:} The entropy of a composite system is additive over the constituent subsystems $(S(E_a)+S(E_b) = S(E_a+E_b))$
    \item \textbf{Continuity and differentiability:} The entropy is a continuous and differentiable function of the extensive parameters. 
    \item \textbf{Extensivity:} The entropy is an extensive function of the extensive variables (not the case when boundary effects are important). $(S(\lambda E, \lambda V)=\lambda S(E,V))$
    \item \textbf{Monotonicity:} The entropy is a monotonically increasing function of the energy for equilibrium values of the energy. 
    \item \textbf{The Nernst postulate:} The entropy of any (real/quantum) system is non-negative. ($\lim_{T\to0} S(T)\geq0$)
\end{enumerate}


\subsubsection*{\boxed{\text{Mathematical identities}}}
\begin{align*}
  N! & \approx N^N e^{-N}\:(\cdot \sqrt{2\pi N}),\quad  \ln{(N!)}  \approx N\ln{N}-N \\
  \int dx\,\delta(u-ax) &= \frac{1}{a}\int dx\,\delta(u/a - x) \\ 
  \int f(x)\delta(g(x))\,dx &= \sum_j \frac{f(x_j)}{g'(x_j)} \\
  \sum_{n=0}^\infty x^n &= \frac{1}{1-x},\quad \lim_{x\to0} x\ln x = 0
\end{align*}


\textbf{Extensivity:}
\begin{align*}
    U&=TS-PV+\mu N \implies 0 = SdT-VdP+Nd\mu \\ 
\end{align*}



\subsubsection*{\scriptsize Standard set of second derivatives}
\begin{align*}
    \alpha &= \frac{1}{V}\left(\pdv{V}{T} \right)_{P,N} - \text{ (th. exp.)},\quad \kappa_T = -\frac{1}{V}\left(\pdv{V}{P} \right)_{T,N} - \text{ (isoth. comp.)} \\ 
    c_{P/V} &= \frac{T}{N}\left(\pdv{S}{T} \right)_{P/V,N} - \text{ (spec. heat pr. part. at const. P/V)} \\
    c_P &= c_V + \frac{\alpha^2 TV}{N\kappa_T},\quad \kappa_S = \kappa_T - \frac{TV \alpha^2}{Nc_P}
\end{align*}


\subsubsection*{\scriptsize Stability Conditions}

\begin{align*}
    (\partial_S^2 U)_{V,N} &= \frac{T}{N c_V} \geq 0,\quad (\partial_V^2 U)_{S,N} = \frac{1}{V\kappa_S}  \geq0 \\
    (\partial_V^2 F)_{T,N} &= \frac{1}{V \kappa_T} \geq 0,\quad (\partial_T^2 F)_{V,N} = -\frac{N}{T}c_V \leq 0 
\end{align*} 
From Legendre trans., different conditions are negative reciprocals of each other, e.g. $(\partial_S^2 U)_{V,N} = - 1 / (\partial_T^2 F)_{V,N}$



\subsubsection*{\scriptsize Phase Transitions}
To illustrate, consider the van der Waals Fluid. From ideal gas, add attraction term, $-a N^2/V$, for neighboring particles, and restrict volume due to hard particle spheres, $V\to V-bN$. This yields 
\begin{align*}
    F_{IG} &= -N k_B T \bracket{\ln(V/N) + \frac{3}{2}\ln(k_B T)+X} \\ 
    F_{VdW} &= - N k_B T \bracket{\ln\closed{\frac{V-bN}{N}} + \frac{3}{2}\ln(k_B T)+X} - a(N^2/V)
\end{align*}

This yields the following expression for pressure and energy 
\begin{align*}
    P = \frac{Nk_B T}{V - bN} - \frac{a N^2}{V^2},\quad U = \frac{3}{2}N k_B T - a \closed{\frac{N^2}{V}}
\end{align*}


PT 
BOLTZMANN+MAX-BOLT 

\subsection*{\boxed{\text{Classical statistical mechanics}}}

\textbf{Microcanonical ensemble:} Assign equal prob. to each microstate, $P_s=1/W$, where $W$ is the number of microstates in energy range.   

\begin{align*}
    \Omega(E,V,N) &=\frac{1}{h^{3N}N!}\int dq \int dp\, \delta(E-H(p,q)) \\
    Z &= \int dE\, \Omega(E,V,N) \exp(-\beta E) = \frac{1}{h^{3N}N!}\int dq \int dp\, e^{-\beta H(q,p)} 
\end{align*}


\subsubsection*{\scriptsize Liouville Theorem}
Systems near region in phase space (ensemble), view as density on grained scale $h^{3N}$. The number of systems in region $(\Q,\P)$ at $t$ is then $\rho(\Q,\P,t)d\Q d\P$. No time dependece for macr. quant. in equil., so const. density. Follow Newton's e.o.m., obey continuity equation 
\begin{align*}
    \pdv{\rho}{t} &= -\vec{\nabla} \cdot (\rho \vec{v}) = -\sum_{\alpha=1}^{3N}\closed{\pdv{(\rho\dot{q}_\alpha)}{q_\alpha} + \pdv{(\rho\dot{p}_\alpha)}{p_\alpha} }
\end{align*} 
Hamilton's equations $\dot{q}_\alpha=\pdv{H}{p_\alpha},\,\dot{p}_\alpha=-\pdv{H}{q_\alpha}$ make the derivatives of the dotted terms cancel, get 
\begin{align*}
    \pdv{\rho}{t} &= -\sum_{\alpha=1}^{3N} \closed{\pdv{\rho}{q_\alpha} \dot{q}_\alpha + \pdv{\rho}{p_\alpha}\dot{p}_\alpha }
\end{align*}  
Change in density found by tot. $t$-der, and is equal to zero (\textbf{Liouville theorem})
\begin{align*}
    \dv{\rho}{t} &= \pdv{\rho}{t} + \sum_{\alpha=1}^{3N} \closed{\pdv{\rho}{q_\alpha} \dot{q}_\alpha + \pdv{\rho}{p_\alpha}\dot{p}_\alpha } = 0
\end{align*}
In equil. SM, $\rho$ itself should be $t$ ind. Want $\pdv{\rho}{t}=0$.  
\begin{align*}
    \pdv{\rho}{t} = \curly{H,\rho},\quad \curly{A,B}=\sum_\alpha \closed{\pdv{A}{q_\alpha}\pdv{B}{p_\alpha} - \pdv{A}{p_\alpha}\pdv{B}{q_\alpha}} 
\end{align*}

If $\curly{H,\rho}=0\implies\pdv{\rho}{t}=0$. If $\rho=\rho(H)$, or $\curly{H,\rho(H,\curly{c_i})}$ where $c_i$ is const. of motion, the Poisson brackets become zero.Thus, the microcan. ens. is const. in time, depends on $H$ only. 
\begin{align*}
    \rho(\Q,\P) = \frac{1}{h^{3N} N!} \delta\closed{E-H(\Q,\P)}
\end{align*}


\subsubsection*{\tiny The ergodic hypothesis}
ESM: ens. avg. $\expval{f}$. Experimentally, measure single system over finite $t$, average $\bar{f}=\frac{1}{T} \int_0^T dt\,f(T)$. Ergodic hypothesis: $\boxed{\expval{f}=\bar{f}}$. May have $\expval{f}\neq\bar{f}$.


\subsection*{\boxed{\text{QM statistical mechanics}}}

\begin{align*}
    P_n &= \frac{e^{-\beta E_n}}{Z} \\
    Z &= \sum_l \Omega(l)\exp(-\beta E_l),\quad \Omega(l)=\text{Degeneracy of energy level}\: l \\
    \closed{\pdv{(\beta F)}{\beta}}_{V,N} &= -\pdv{\ln Z}{\beta} \implies \beta F = -\ln Z + f(V,N) \\
    S &= -k_B \sum_n P_n \ln P_n
\end{align*}


For identical particles $f(V,N)=0$, for distinguishable particles $f(V,N)=-\ln N!$. 


\subsubsection*{\scriptsize Third law of TD}
Let $l=0$ be the lowest energy level. The partition function can be written as 
\begin{align*}
    Z &= \Omega(0)e^{-\beta E_0} \closed{1 + \sum_{l>0} \frac{\Omega(l)}{\Omega(0)} e^{-\beta(E_l-E_0)} } \to \Omega(0)e^{-\beta E_0}   \\
    P_n \to P_0 &= \frac{1}{\Omega(0)} \implies S = k_B \ln \Omega(0) = \text{const.} 
\end{align*}

Switching sums and products 
\begin{align*}
    Z &= \sum_{\curly{n_j}} \prod_{j=1}^N \exp(-\beta E_{n_j}) = \prod_{j=1}^N \left( \sum_{n_j} \exp(-\beta E_{n_j}) \right)
\end{align*}


Energies and density of states for 3D quantum ideal gas. 
\begin{align*}
    \epsilon_{\vec{k}} &= \frac{\hbar^2 }{2m}k^2 = \frac{\hbar^2 \pi^2}{2mL^2}n^2 = \epsilon_{\vec{k}} \\ 
    D(\epsilon) &= \int_0^\infty dn_x\,dn_y\,dn_z \delta(\epsilon-\epsilon_{\vec{n}}) = \frac{V}{4\pi^2} \left(\frac{2m}{\hbar^2}\right)^{3/2} \epsilon^{1/2}
\end{align*}



\subsubsection*{\scriptsize Grand Canonical Ensemble}
In equilibrium with a reservoir. Can exchange energy and particles. 
\begin{align*}
    P(E,N) &= \frac{1}{\mathcal{Z}}\Omega(E,V,N) e^{-\beta (E - \mu N)},\quad \mathcal{Z} = \sum_{N=0}^\infty \sum_E \Omega e^{-\beta (E - \mu N)}
\end{align*}

Denote QM state by $\alpha$, occupation number $n_\alpha$ ($n_\alpha=0,1$ for e.g. electrons). 
\begin{align*}
    \mathcal{Z} &= \sum_{\{n_\epsilon\}} \prod_\epsilon e^{-\beta(\epsilon-\mu)n_\epsilon} = \prod_\epsilon \sum_{n_\epsilon} e^{-\beta(\epsilon-\mu)n_\epsilon} = \prod_\epsilon \mathcal{Z}_\epsilon \\
    \expval{n_\epsilon} &= \frac{1}{\mathcal{Z}_\epsilon} \sum_{n_\epsilon} n_\epsilon e^{-\beta(\epsilon-\mu)n_\epsilon},\quad \expval{N} = \sum_\epsilon \expval{n_\epsilon},\quad U = \expval{E} = \sum_\eps \eps \expval{n_\eps}
\end{align*} 



\subsubsection*{\scriptsize Bosons and Fermions}
Using $+$ for fermions and $-$ for bosons: 
\begin{align*}
    \expval{n_\eps} &= \frac{1}{e^{\beta(\eps-\mu)} \pm 1},\quad \mathcal{Z} = \begin{cases}
        \prod_\eps (1+\exp(-\beta(\eps-\mu))),\:&\textbf{f} \\ 
        \prod_\eps {1-\exp[-\beta(\eps-\mu)]}^{-1},\:&\textbf{b}
    \end{cases} \\
    \ln \mathcal{Z} &= \pm \sum_\eps \ln(1 \pm e^{-\beta(\eps-\mu)}) \approx \pm \int_0^\infty d\eps\, D(\eps) \ln(1 \pm e^{-\beta(\eps-\mu)}) \\
    N &= \int_0^\infty d\eps\,D(\eps) (\exp[\beta(\eps-\mu)]\pm1)^{-1} \\ 
    U &= \int_0^\infty d\eps\, \eps D(\eps) (\exp[\beta(\eps-\mu)]\pm1)^{-1} \\ 
    \ln \mathcal{Z} &= \beta PV \:(\text{for extensive system})
\end{align*}



\subsubsection*{\scriptsize Bose-Einstein statistics}
Since $\expval{n_\eps}>0$, must have $\eps>\mu$ for bosons. Set lowest energy state as $\eps=0\implies\mu<0$. At low-T, using $x=\beta\eps$, $D(\eps)=\chi \eps^{1/2}$ and $e^{\beta\mu}=\lambda$ 
\begin{align*}
    N = \chi (k_B T)^{3/2} \int_0^\infty dx\, \frac{x^{1/2}}{\lambda^{-1} e^x - 1}
\end{align*}
Since $\mu<0\implies\lambda^{-1}>1$. $N$ should be const., but decreases by $T^{3/2}$ for red. $T$. Need $\lambda^{-1}$ small, but $\lambda^{-1}=1$ is the absolute limit. At $\lambda=1$, the integral is 
\begin{align*}
    N = \chi (k_B T_E)^{3/2} 2.315 \implies k_B T_E = \left( \frac{2\pi\hbar^2}{m}\right) \left( \frac{N}{2.612V} \right)^{2/3}
\end{align*} 
$N$ expression invalid for $T<T_E$. Can't approximating the sum as an integral at low $T$, since ground state is heavily occupied and the summand unsmooth. Treat lowest energy level separate. 
\begin{align*}
    N &= N_0 + \int_{\eta\to0^+}^\infty d\eps\,D(\eps) \expval{n_\eps},\quad\text{Error goes to zero as } \eta\to0 \\
    &= N_0 + N \left(\frac{T}{T_E}\right)^{3/2} = [\exp(-\beta\mu)-1]^{-1} = N\bracket{1-\closed{\frac{T}{T_E}}^{3/2} } \\
    \mu &\approx -\frac{k_B T}{N}\bracket{1-\closed{\frac{T}{T_E}}^{3/2}}^{-1},\quad\text{expanding small }\beta\mu \text{ for } T<T_E
\end{align*}


\subsubsection*{\scriptsize Fermi-Dirac statistics}
The occupation number as $T\to0$ is (MULTIPLY D BY FACTOR 2 FOR ELECTRONS) 
\begin{align*}
    f(\eps)&=\frac{1}{e^{\beta(\eps-\mu)} +1} \xrightarrow{T\to0} \Theta(\eps_F-\eps),\quad \eps_F\equiv \lim_{T\to0}\mu(T,N) \\
    \implies N&=\sum_{\vec{k}} f(\eps_{\vec{k}}) = \int_0^{\eps_F} d\eps\, D(\eps) = X \frac{2}{3} \eps_F^{3/2},\quad U=X \frac{2}{5} \eps_F^{5/2} \\
    \implies \eps_F & \propto (N/V)^{2/3}, \quad \text{since } X\propto V \\
    U/N &= \frac{3}{5}\eps_F \xrightarrow[T=0]{\text{Euler eq.}} PV=\frac{2}{5}\eps_F N \\ 
    \eps_F &= y(N/V)^{2/3}\implies \kappa_T^{-1}= \frac{2}{3}\eps_F \frac{N}{V}
\end{align*}


\subsubsection*{\scriptsize Sommerfeld expansion}
At low non-zero $T$, valid for $k_B T/\eps_F\ll1$. 
\begin{align*}
    I &= \int_0^\infty d\eps\, \phi(\eps) f(\eps) \\    
    f(\mu+x) &= \frac{1}{e^{\beta x}+1} = 1 - \frac{1}{e^{-\beta x}+1} = 1-f(\mu-x) \\
    I &=\int_0^\mu d\eps\, \phi(\eps) - \int_0^\mu d\eps\, \phi(\eps)\frac{1}{e^{-\beta(\eps-\mu)} +1 } + \int_\mu^\infty d\eps\, \phi(\eps) \frac{1}{e^{\beta(\eps-\mu)} +1}
\end{align*}

The first term is the step func. contr., the latter two are corrections. Substitute $z=-\beta(\eps-\mu)$ and $z=\beta(\eps-\mu)$ for the two corr. terms, respectively. Approx. $z=\beta\mu\to\infty$.  
\begin{align*}
    &\phi(\mu+z/\beta) - \phi(\mu-z/\beta) = \frac{2z}{\beta}\phi'(\mu) + \frac{2}{3!}\closed{\frac{z}{\beta}}^3 \phi'''(\mu)+... \\
    I&=\int_0^\mu d\eps\, \phi(\eps) + \int_0^\infty \frac{d\eps}{\beta} \frac{\phi(\mu+z/\beta) - \phi(\mu-z/\beta)}{1+e^z} \\
     &= \int_0^\mu d\eps\, \phi(\eps) + (k_B T)^2 \phi'(\mu) 2 \int_{0}^\infty dz\, \frac{z}{e^z+1} + (k_B T)^4 \phi'''(\mu) \frac{2}{3!} \int_0^\infty dz\, \frac{z^3}{e^z+1} \\ 
    &= \int_0^\mu d\eps\, \phi(\eps) + (k_B T)^2 \phi'(\mu) \frac{\pi^2}{6} + (k_B T)^4 \phi'''(\mu) 7 \frac{\pi^4}{360} + \mathcal{O}(T^6)
\end{align*} 


\begin{align*}
    U:\:\phi(\eps)=X\eps^{3/2}\implies U=X[2/5 \mu^{5/2} + \pi^2/4 (k_B T)^2 \mu^{1/2}] + \mathcal{O}(T^4)
\end{align*}

Need $\mu$, $N(T=0)$ known, should be const. for incr. $T$. 
\begin{align*}
    N &= X \frac{2}{3}\eps_F^{3/2} = X \frac{2}{3}\mu^{3/2} + X\frac{\pi^2}{12} (k_B T)^{2} \mu^{-1/2} + \mathcal{O}(T^4) \\ 
    \implies \eps_F^{3/2} &= \mu^{3/2}\bracket{1+\frac{\pi^2}{8}\closed{\frac{k_B T}{\mu}}^2 } + \mathcal{O}(T^4)
\end{align*}
$(k_B T/\mu)^2$ is small and can be replaced by $(k_B T/\eps_F)^2$ with and error $\mathcal{O}(T^4)$, and is solved by iteration. Solving the above equation for $\mu$ and taylor expanding the term of power $(-2/3)$ yields 
\begin{align*}
    \mu \approx \eps_F \closed{1 - \frac{\pi^2}{12} \closed{\frac{k_B T}{\eps_F}}^2+...}
\end{align*}

Plugging in for $U$ and expanding $\mu^{5/2}$ and $\mu^{1/2}$ up to $T^2$ gives 
\begin{align*}
    U &= \frac{2}{5}X\eps_F^{5/2} + \frac{\pi^2}{6}(k_B T)^2 X\eps_F^{1/2} \implies C_V = \frac{\pi^2}{2} Nk_B \closed{\frac{k_B T}{\eps_F}} + \mathcal{O}(T^3)
\end{align*}
where $X\eps_F^{3/2}=3/2\cdot N $. The linear dependence is observed for metals at low $T$. 


\subsubsection*{\scriptsize Semiconductors - Fermions at low temperature}

Periodic solids have energy bands separated by energy gaps where $D(\eps_V<\eps<\eps_C)=0$, where $\eps_V$ is the upper energy limit of the \textit{valence band} (VB) and $\eps_C$ is the lower energy of the \textit{conductor band} (CB). At $T=0$ electrons fill up every state up to $\eps_F$, and where $\eps_F$ is found is strongly affecting the behavior of materials. Typical metals have the fermi energy in the CB. But, for a full VB and empty CB we need a high energy to excite electrons (Insulator or semiconductor for small gap). The specific heat is then not linear in $T$ as we found for $T=0$ in the Sommerfeld expansion. 


The number of electrons is given by 
\begin{align*}
    N &= \int_0^{\eps_V} d\eps\,D_V(\eps) f(\eps) + \int_{\eps_C}^\infty d\eps\,D_C(\eps) f(\eps) \\ 
    &= \int_0^{\eps_V} d\eps\,D_V(\eps)\quad\text{for low}\: T 
\end{align*}
since $f(\eps<\mu)=1$ and $f(\eps>\mu)=0$ as $T\to 0$. Want the same $N$ as we increase $T$. Subtracting the two terms yield 
\begin{align*}
    0 &= \int_0^{\eps_V} d\eps\,D_V(\eps) [f(\eps)-1] + \int_{\eps_C}^\infty d\eps\,D_C(\eps) f(\eps)
\end{align*}

At small, finite $T$, assuming $\mu$ in the middle of the gap gives $f(\eps)-1=-1/(e^{-\beta(\eps-\mu)}+1)\approx-e^{\beta(\eps-\mu)}$ for $\eps<\eps_V$ and $f(\eps)\approx e^{-\beta(\eps-\mu)}$ for $\eps>\eps_C$. Setting $x=\beta(\eps-\eps_C)$ and $y=\beta(\eps_V-\eps)$ gives 
\begin{align*}
    0 &= \int_{\eps_C}^\infty d\eps\,D_C(\eps) e^{-\beta(\eps-\mu)} - \int_0^{\eps_V} d\eps\,D_V(\eps) e^{\beta(\eps-\mu)} \\
    &= \int_0^\infty \frac{dx}{\beta} D_C(x/\beta+\eps_C) e^{\beta(\mu-\eps_C)-x} + \int_{\beta\eps_V\to\infty}^0 \frac{dy}{\beta} D_V(\eps_V-y/\beta) e^{\beta(\eps_V-\mu)-y} 
\end{align*}
With a factor $e^{\beta\mu}$ in both integrals, we get the ratio 
\begin{align*}
    e^{2\beta\mu} &= e^{\beta(\eps_V+\eps_C)} \bracket{\frac{\int_0^\infty \frac{dy}{\beta} D_V(\eps_V-y/\beta) e^{-y} }{\int_0^\infty \frac{dx}{\beta} D_C(x/\beta+\eps_C) e^{-x}}}
\end{align*}
Taking the log to obtain $\mu$ expression, we assume $D_C(\eps)=A(\eps-\eps_C)^a$ for $\eps>\eps_C$ and $D_V(\eps)=B(\eps_V-\eps)^b$ for $\eps<\eps_V$. Then, 
\begin{align*}
    \mu &= \frac{\eps_V+\eps_C}{2} + \frac{1}{2\beta} \ln((k_B T)^{b-a} X) \\
    &= \frac{\eps_V+\eps_C}{2} + \frac{b-a}{2} k_B T \ln k_B T +\frac{1}{2}k_B T \ln X
\end{align*}
where X is the remaining integral expression that is independent of $T$. For $a=b$ we get 
\begin{align*}
    \mu = \frac{\eps_V+\eps_C}{2} + \frac{1}{2}k_B T \ln(B/A)
\end{align*}
which is linear in $T$. 

Using the expression for $\mu$ with $A\sim B$ at low $T$ one find $N_C \sim e^{-\beta/2 (\eps_C-\eps_V)}$ and is exponentially suppressed at low $T$. 


\subsubsection*{\tiny Extrinsic Semiconductors (adding dopants)}
Add impurity in the gap just below $\eps_C$. Two kind of states:
\begin{itemize}
    \item Bond state: b,s = band number and $\vec{k}$, $\curly{\uparrow,\downarrow}$  
    \item Donor state: d,s = where the state is, $\curly{\uparrow,\downarrow}$
\end{itemize}
Cost much energy to have both up and down in the same donor state (Coulomb). Ignore Coulomb in overall model, but ommit two spins in a single donor state. The partition function is $\Z=\Z_b \Z_d$. For $\Z_b$ we use $\eps_{b,s}=\eps_b$

\begin{align*}
    \Z_b &= \prod_b \closed{\sum_{n_{b,\uparrow}} e^{-\beta(\eps_b-\mu)n_{b,\uparrow}} } \closed{\sum_{n_{b,\downarrow}} e^{-\beta(\eps_b-\mu)n_{b,\downarrow}} } \\ 
    &= \prod_b \closed{1 + 2 e^{-\beta(\eps_b-\mu)} + e^{-2\beta(\eps_b-\mu)}}
\end{align*}

For donor levels, we get 
\begin{align*}
    \Z_d &= \prod_d \Z_{d,1} =  \prod_d (1+2\exp(-\beta\eps_d - \mu)) \\ 
    n_d &= \frac{1}{\frac{1}{2}e^{\beta(\eps_d-\mu)}+1}
\end{align*} 
The factor $1/2$ in $n_d$ for donor levels has given the name \textit{semiconductor statistics} to the occupation number. 



\subsubsection*{The Harmonic solid}
1D crystal lattice with spacing $a$. Pos.: $r_j = R_j + x_j$, $R_j=a\cdot j$ is equil. pos., $x_j$ is the deviation and $j=0,1,\dots,N-1$. Model as springs, with P.BC, 
\begin{align*}
    H = \frac{m}{2}\sum_{j=0}^{N-1} \dot{x}_j^2 + \frac{K}{2} \sum_j (x_{j+1}-x_j)^2
\end{align*}

Introduce Fourier transform to remove couplings.
\begin{align*}
    X_k = \frac{1}{\sqrt{N}} \sum_j x_j e^{-ikR_j},\quad x_j = \frac{1}{\sqrt{N}} \sum_k X_k e^{ikR_j},\quad x_j\in\mathbb{R}\implies X_k^* = X_{-k}
\end{align*}

P.BC: $x_{j+N}=x_j\implies k = \frac{2\pi}{Na}n$, $n\in\mathbb{Z}$. Also, for $\tilde{k}=2\pi z/a$, $z\int\mathbb{Z}$, $X_{\tilde{k}}=X_k$. All info about $x_j$ gotten from $X_k$ in $k\in[-\pi/a,\pi/a)$. This is the \textit{First Brillouin zone}. Then 
\begin{align*}
    n = 0,\pm1,\pm2,\dots, \begin{cases}
        \pm(N-1)/2 \quad&\text{for }\: N\: \text{odd} \\ 
        \pm(N/2-1),N/2\quad&\text{for }\: N\: \text{even}
    \end{cases}
\end{align*}


Now, the kinetic energy can be expressed through the fourier modes, using a geometric series 
\begin{align*}
    \sum_j \dot{x}_j^2 &= \sum_j \frac{1}{N}\sum_{k,k'}\dot{X}_k \dot{X_k'} e^{i(k+k')R_j} \\ 
    \sum_{j=0}^{N-1} e^{i(k+k')R_j} &= N\delta_{n,-n'} \implies \sum_j \dot{x}_j^2 =  \sum_k \abs{\dot{X}_k}^2 
\end{align*}
Doing a similar calculation for the potential term, one obtains eventually 
\begin{align*}
    H = \frac{m}{2} \sum_k \abs{\dot{X}_k}^2 + \frac{K}{2} \sum_k 4\sin^2(ka/2) \abs{X_k}^2
\end{align*}

$H$ is now a collection of independent harmonic oscillators with stiffness $K_k$ and frequency, $\omega_k$ with 
\begin{align*}
    K_k = 4K \sin^2(ka/2),\quad \omega_k^2 &= K_k/m = \tilde{\omega}^2 4\sin^2(ka/2) \\
\end{align*}

The harmonic solid is essentially a sum of indipendent QM harmonic oscillators. 

The partition function is a product over $k$, since $H$ is a sum over $k$. It is also a sum over occupation number $n$, and thus yields 
\begin{align*}
    Z &= \prod_k \sum_{n=0}^\infty e^{-\beta \hbar \omega_k (n+1/2)},\: F= \sum_k \closed{\frac{\hbar \omega_k}{2} + k_B T \ln\closed{1-e^{-\beta \hbar \omega_k}} } \\
    U &= \sum_k \closed{\frac{\hbar \omega_k}{2} + \frac{\hbar \omega_k}{e^{\beta \hbar \omega_k} - 1} } \\ 
    U &= \frac{Na}{2\pi} \int_{-\pi/a}^{\pi/a} dk\, \closed{\frac{1}{2}\hbar \omega_k + \frac{\hbar \omega_k}{e^{\beta \hbar \omega_k} - 1} } \approx \frac{Na}{2\pi} \int_{-\pi/a}^{\pi/a} dk\, \closed{\frac{1}{2}\hbar \omega_k + k_B T} \\ 
\end{align*}
Last equality valid for large $N$ and high $T$, respectively. High $T:\:c_V=d\cdot k_B$.  

\subsubsection*{Debye approximation}
For $\beta\to\infty$ second term in the $U$ integral dies, but $\omega_k\to0$ for certain $k$. Only low-energy modes sign. excited at low-$T$ (higher ones exp. suppressed).  

$\omega_k\approx \tilde{\omega} ka$ for small $k$. \textit{Debye approximation:} $\hbar\omega_k=\hbar v \abs{\vec{k}}$ as lin. rel., $v$ is speed of sound (interpolate between known low and high $T$ sol).  

3D, approximate 1st BZ by sphere of radius $k_D$ so it contains $3N$ modes. Have $n_x,n_y,n_z$ positive with $3$ modes each. 
\begin{align*}
    3N&=\frac{3}{8}\int_0^{n_D} 4\pi n^2 dn \implies n_D = (6N/\pi)^{1/3},\: k_D = \closed{\frac{6\pi^2}{a^3}}\quad \text{No N for Olav}
\end{align*}
Debye energy: $\hbar \omega_D\equiv \hbar v k_D$. \textit{Debye temperature:} $\theta_D\equiv \hbar \omega_D/k_B$. Using spherical coordinates and then setting $x=\beta\hbar v k$, we get 
\begin{align*}
    U &= \text{const} + 3N\closed{\frac{a}{2\pi}}^3 \int_{\text{1st B.Z}} d^3 k\,\frac{\hbar v \abs{\vec{k}}}{e^{\beta\hbar v \abs{\vec{k}}} - 1}\:(\text{gen. HS in 3D}) \\ 
    &= \text{const} + 9N \frac{k_B T}{(\theta_D / T)^3} \int_0^{\theta_D / T} dx\, \frac{x^3}{e^x - 1}
\end{align*}
For high $T\:(T\gg\theta_D)$ we have $x<1$ and we get 
\begin{align*}
    U &= \text{const} + 9N \frac{k_B T}{(\theta_D / T)^3} \int_0^{\theta_D / T} dx\, x^2 = \text{const} + 3N k_B T
\end{align*}
At $T\ll\theta_D$, approximate upper limit as $\theta_D/T\to\infty$. Integral becomes $\pi^4/15$. 
\begin{align*}
    U &= \text{const} + \frac{3N\pi^4}{5} \frac{(k_B T)^4}{(\hbar\omega_D)^3},\: c_V = \frac{12}{5}\pi^4 k_B \closed{\frac{k_B T}{\hbar\omega_D}} \propto T^3 
\end{align*}
which is the same $T$ dependency as for insulators, and is observed to be correct for insulating crystals. Approx excellent at high and low $T$. Spec. heat with $T^3$-dependence at low $T$, correct for lattice vib in real crystals, and const. $3k_B$ at high $T$ (Dulong and Petit).

\textbf{Summary of the Debye approximation:} 1 - Replace true energy spectrum with approx that is linear in $k$, and sph-symm., $\eps(\vec{k})=\hbar v \abs{\vec{k}}$. 2 - Replace true B.Z. by sph-symm. 3 - Choose size of B.Z so it contains exactly $N$ $k$'s and $3N$ modes. 



\subsubsection*{\boxed{\text{Ising model}}}
\begin{align*}
    H = -J \sum_{\langle{i,j}\rangle} \sigma_i \sigma_j - h \sum_i \sigma_i 
\end{align*}

\subsubsection*{\scriptsize Ising chain (1D)}
Set $h=0$. $H = -J\sum_{i=1}^{N-1} \sigma_i \sigma_{i+1}$ Use O.BC, define $\tau_1=\sigma_1,\,\tau_i=\sigma_{i-1}\sigma_i$, $\tau_i=\pm1$. Then $H=-J\sum_{i=2}^N \tau_i$, independent of $\tau_1$ (invar. for all spins fliped). 

\begin{align*}
    Z &= \sum_{\tau_1} \prod_{i=2}^N \closed{\sum_{\tau_i} e^{\beta J \tau_i}} = 2(2\cosh(\beta J))^{N-1} \\ 
    U &= -(N-1)J\tanh(\beta J),\quad c = k_B \beta^2 J^2 \closed{1-\frac{1}{N}} \frac{1}{\cosh^2(\beta J)}
\end{align*} 


\subsubsection*{\scriptsize Ising chain with transfer matrices}
$J\neq0$, $h\neq0$, use P.BC. ($\sigma_{N+1}=\sigma_1$). Write H symmetrically 
\begin{align*}
    H &= -J\sum_{i=1}^N \sigma_i \sigma_{i+1} - \frac{h}{2}\sum_{i=1}^N (\sigma_i+\sigma_{i+1}) \\
    Z &= \sum_{\curly{\sigma}} \prod_{i=1}^N T_{\sigma_i,\sigma_{i+1}},\quad T = \begin{pmatrix}
        e^{\beta J-\beta h} & e^{-\beta J} \\ 
        e^{-\beta J} & e^{\beta J + \beta h}
    \end{pmatrix} \\
    Z &= \sum_{\curly{\sigma}} T_{\sigma_1,\sigma_2} \dots T_{\sigma_N,\sigma_1} = \Tr\closed{T^N} = \lambda_1^N + \lambda_2^N
\end{align*}
For large $N$, let $\lambda_1>\lambda_2$, so $Z = \lambda_1^N$
\begin{align*}
    \lambda_1 &= e^{\beta J}\cosh(\beta J) + \sqrt{e^{2\beta J}\cosh^2(\beta h)-2\sinh(2\beta J) } \\ 
    \frac{F}{N} &=-k_B T \ln\lambda_1 \xrightarrow{h=0} -k_B T \ln(2\cosh(\beta J))
\end{align*}
for $h=0$, we get $F/N=-k_B T \ln(2\cosh(\beta J))$. There is no phase transition in $1D$. 

The magnetization, $m$, and magnetic susceptibility, $\chi$, is given by 
\begin{align*}
    m &= \frac{1}{N} \expval{\sigma_j} = \frac{1}{\beta N} \pdv{\ln Z}{h} = -\frac{1}{N}\pdv{F}{h},\quad \chi = \pdv{m}{h} = \frac{1}{\beta N} \pdv[2]{\ln Z}{h}
\end{align*}



\subsubsection*{Mean Field Approximation}
Approximate $m=1/N \sum_j \expval{\sigma_j}=m_j$, i.e. all spins have the same average. Define the deviation from the average $\tilde{\sigma}_j=\sigma_j-m$, assuming that it's small. The interaction term now becomes 
\begin{align*}
    \sigma_j\sigma_k  \approx m^2 + m(\tilde{\sigma}_j + \tilde{\sigma}_k) \implies \sigma_j \sigma_k \approx -m^2 + m(\sigma_j + \sigma_k)
\end{align*}
In the Hamiltonian, include factor $1/2$ for the double counting of spin. Sum over $\sigma_j \sigma_{j+\delta}$. Introduce $z=2d=\sum_\delta 1$. PBC: Shift $j\to j'=j+\delta$ 
\begin{align*}
    H &= -\frac{J}{2}\sum_j \sum_\delta \closed{-m^2 + m\sigma_j + m\sigma_{j+\delta}} = Jm^2 \frac{Nz}{2} - J mz \sum_j \sigma_j
\end{align*}  

Adding the field again, defining $h_\mathrm{eff}=Jmz+h$, we get $H_\mathrm{MFA}=Jm^2 \frac{Nz}{2} - h_{\mathrm{eff}}$. Since all spins are the same on average, sum over $\sigma_1$ $N$ times. The partition function now becomes 
\begin{align*}
    Z &= e^{-\beta J m^2 Nz/2} \closed{2\cosh(\beta h_{\mathrm{eff}})}^N = e^{-\beta J m^2 Nz/2} Z_1^N
\end{align*}


Now, we can calculate the average magnetization, remembering that $m=m_j\implies m=m_1=\expval{\sigma_1}$, so we get 
\begin{align*}
    m &= \frac{1}{Z_1} \sum_{\sigma_1=-1}^{+1} \sigma_1 e^{(\beta h_{\mathrm{eff}}\sigma_1)} = \tanh(\beta h + \beta Jzm)
\end{align*}
where the last equality is the MFA self-consistent equation for $m$, but there is no closed form solution for it. For $h=0$, set $x=\beta J zm\implies \frac{k_B T}{Jz}x=\tanh(x)$. For disordered we may have the LHS bigger than $1$, which means that $m=0$. For the ordered phase, $k_B T/Jz < 1$, there are $3$ solutions, $x=\pm1,0$, but $x=0$ turns out to be unstable. The critical temperature is $k_B T_c=zJ=4J$ for the square lattice. The exact value, however is $k_B T_c\approx 2.26 J$.   

The unstable solution of $m=0$ in the ordered phase is due to the fact that $F$ must be minimized to be stable, while $k_B T/Jz < 1$ corresponds to a maxima of $F$ for $m=0$.  


For $T\lesssim T_c$ with $h=0$, $m$ is small, and we can expand to solve for $m$
\begin{align*}
    m \approx \beta J m z - \frac{1}{3} (\beta m J z)^3 \implies m^2 = 3 \closed{\frac{T}{T_c}}^2 \closed{1-\frac{T}{T_c}} 
\end{align*}
where we have used that $Jz=k_B T_c$. Near $T_c$, the squared term is negligible, and we get 
\begin{align*}
    m \propto \closed{\frac{T_c-T}{T_c}}^{1/2}
\end{align*}

We always have $m\propto(...)^{\beta}$, and for MFA $\beta=1/2$, while the 2D Ising model has $\beta=1/8$ in reality. 

\


\subsubsection*{\scriptsize Phase Transitions (PT) - Generalities}
\textbf{Definitions:}
\begin{itemize}
    \item \textbf{Phase:} Region in phase space where free energy is analytic (function of its arguments)
    \item \textbf{Phase transition:} Non-analytic
    \item \textbf{First order PT:} The derivative of the free energy is discontinuous. 
    \item \textbf{Continuous PT:} All first derivatives of free energy is continuous, but one or more higher order derivatives are discontinuous. 
\end{itemize}

\subsubsection*{\tiny Existence of PT} 

\textbf{Energy/Entropy argument:} THe partition function is $Z=\exp(-\beta F)=\sum_s \exp[-\beta H(s)]$, and at low $T$, $Z$ is dominated by the lowest energy configurations. Assume there are $N_g$ such states with energy $E_g$. This yields $Z\approx N_g \exp(-\beta E_g)=\exp[-\beta(E_g - k_B T \ln N_g)]$. Approximating the entropy as $S\approx k_B T \ln N_g$ we get $F_g(\text{low}-T)\approx E_g - k_B T \ln N_g$. $N_g$ is usually small. Excited states have $E>E_g$ and are not favored at low $T$, unless there are considerably many of them, such that their entropy becomes significant. Denote excitation by $E_{\mathrm{ex}}$, which is not much larger than $E_g$, with $N_\mathrm{ex}$ so that $F_\mathrm{ex}=E_\mathrm{ex}-k_B T \ln N_\mathrm{ex}$. We expect a PT when $F_g=F_\mathrm{ex}$. For small $\ln N_g$ we then get 
\begin{align*}
    k_B T_c \approx \frac{\Delta E}{\ln N_\mathrm{ex}}
\end{align*} 
We expect excited states to dominate $Z$ for $T>T_c$, and $E_g$ states to dominate for $T<T_c$. 

For the 1D ising model of $N$ spins with open boundary conditions and no applied magnetic field, there are $N_g=2$ ground states (all up or all down), with $E_g=-J(N-1)$. The first excited states is with approximately half of spins up and the other half down, with $E_\mathrm{ex}=E_g+2J$ and $N_\mathrm{ex}=N-1$. We then get 
\begin{align*}
    k_B T_c = 2J/(\ln(N-1)-\ln 2) \to 0,\quad\text{for}\: N\to\infty
\end{align*}
Hence, no PT at finite $T$ in 1D Ising. 

For the 2D Ising model, we have $N_g=2$ again. The energy difference is $\Delta E=2J\cdot l$, where $l$ is the circumference of the domain enclosing a collection of oppositely directed spins. The number of excited states are given by the number of ways of doing a random walk on the lattice with $l$ steps that encloses a domain, and is approximately $N_\mathrm{ex}\approx N(z-1)^l \cdot 2$, where $N$ contribution is from the number of starting points, and $z=4$ is the number of neighbors in 2D lattice. Thus,
\begin{align*}
    k_B T_c = \frac{2J\cdot l}{\ln(2N(z-1)^l)-\ln2} = \frac{2J \cdot l}{\ln N + l\cdot \ln3}
\end{align*}
For large $N$ there are also large $l$-s dominating the domain, so that 
\begin{align*}
    k_B T_c\sim \frac{2Jl}{l\ln(z-1)}=\frac{2J}{\ln3}=1.82 J
\end{align*}
the exact value for 2D Ising is $k_B T_c\approx 2.269J$.

\subsubsection*{\tiny Landau argument for PT existence}
A symmetry can't be continuously deformed into another symmetry. Thus: Two phases with different symmetries are always separated by one or more PT's (There can still be PT between phases of same symmetries).


\subsubsection*{\scriptsize Critical exponents}
Many quantities behave like power laws of $t$ close to $T_c$ for cont. PT's. E.g. 
\begin{align*}
    C(r) = \expval{(m(r) - \expval{m(r)})(m(0) - \expval{m(0)})} \sim f(r) e^{-r/ \xi} 
\end{align*}
And with $\xi\to\infty$ at $T_c$, we get $C(r)\to f(r)$. 

General parameters 
\begin{alignat*}
    \alpha \alpha: &&\quad c_V &\sim \frac{1}{\abs{t}^\alpha} \\ 
    \beta:&& \quad m &\sim (-t)^\beta\:(\text{order param. } t<0) \\ 
    \gamma:&& \quad \chi=\pdv{m}{H}\eval_{H=0} &\sim \frac{1}{\abs{t}^\gamma}\:(t=0) \\ 
    \delta:&& \quad m &\sim \abs{H}^{1/\delta}\: \text{order param. dep. on field (t=0)} \\
    \nu:&& \quad \xi &\sim \frac{1}{\abs{t}^\nu} \\ 
    \eta:&&\quad C(r) &\sim \frac{1}{\abs{r}^{d-2+\eta}}\:\text{corr func}\: (r\ll\xi)
\end{alignat*}

\textbf{Scaling laws:}
\begin{align*}
    \nu(2-\eta) &= \gamma \\
    \alpha + 2\beta + \gamma &= 2 \\
    \beta(\delta-1) &= \gamma \\ 
    2-\alpha &= \nu d
\end{align*}


\subsubsection*{\scriptsize Renomrmalization group}
Universality: Same behavior at PT for several different microscopic systems. 

RG-trans: Trans. betw. different microscopic models behaving the same at macroscopic scales. 

Example: 1D Ising with a const., $C$. 
\begin{align*}
    Z = \sum_{\curly{\sigma}} T_{s_a s_{2a}}\dots T_{s_{Na} s_a}. \quad T_{i=j}=e^{\beta J + \beta C},\:T_{i\neq j}=e^{-\beta J +\beta C}
\end{align*}
Now, perform sum over every second site. Sum over $s_{2a}$ yields $T^2=2e^{2\beta C} \binom{\cosh(2\beta J)\quad 1}{1\quad \cosh(2\beta J)}=e^{\beta C'} \exp\closed{\beta J'\binom{1 \quad -1}{-1 \quad 1}}=T'$. The new transfer matrix is for the Ising chain with spin on every second site. Now, $H'=-\sum_r (J' s_r s_{r+a'}+C'),\:a'=2a$, with 
\begin{align*}
    &2e^{2\beta C} \cosh(2\beta J) = e^{\beta C'} e^{\beta J'},\quad 2e^{2\beta C}=e^{\beta C'} e^{-\beta J'} \\ 
    &K_1 \equiv e^{-2\beta J},\quad K_1' \equiv e^{-2\beta J'},\quad K_2 \equiv e^{-2\beta C},\quad K_2' \equiv e^{-2\beta C'}
\end{align*} 
Now, $K_1=1\implies \beta J\to 0$ (disordered) and $K_1\implies \beta J\to\infty$ (ordered). Solving for $K_1'$, we get $K_1'=\frac{2K_1}{1+K_1^2}$. Since $K_1\in(0,1)$ then $1+K_1^2>1$. Iterating (subsequent RG trans.) for $K_1>0$ we will converge to $K_1=1$. Then, for $K_1\gtrsim0,\:K_1'\approx 2K_1$. We have $K_1=K_1(a),\: K_1'=K_1(2a)$. 

In general, for incr. latt. spacing $s>1$, we have $K(sa)=s^{y_k} K(a)$. $y_k$ is scaling expo., repr. repulsive fp when positive (relevant), attractive fp when negative (irrelevant) (marginal for $y_k=0$, go beyond lin.). 

The scaling relation in general is, for any dim-full quantity, $Q$ with satial dim $D$, measured in units off latt. spacing, $\boxed{\tilde{Q}(\curly{K})=s^D \tilde{Q}(\curly{K s^{y_k}})}$.

Usually two relevant coupling, $t$ and $h$. Drop hats $\xi(h,t) = s \xi(hs^{y_h},ts^{y_t}),\:y_h,y_t>0$. arbitrary $s>1$, choose it s.t. $ts^{y_t}=1\implies s=t^{-1/y_t}$. 
\begin{align*}
    \xi(h,t) = t^{-1/y_t} \xi(ht^{-y_h / y_t},1),\:h=0\implies \xi(0,t) = \frac{1}{t^{1/y_t}} \xi(0,1)
\end{align*}
where $x(0,1)$ is a number. Compare to crit. exp. we see that $\xi\sim 1/\abs{t^\nu}\implies \nu=1/y_t$. For $m=-T df(h=0)/dh$, choose $s=(-t)^{-1/y_t}$. 

\subsubsection*{\tiny Finite size scaling}
Numerics: Finite system size, $L$. Intr. dimless len $L^{-1}=(L/a)^{-1}$, for $a'=as$ dimless len incr. as $L'^{-1}=sL^{-1}$. Then, corr.len (drop $h$) scales as (choose $s=L$)
\begin{align*}
    \xi(t,L^{-1}) = L \xi(t L^{y_t},1)=L g(tL^{y_t}),\quad \xi(tL^{y_t}\to\infty)\sim 1/t^\nu \implies g(x)\to 1/x^{1/y_t} \\
    \text{(Fin.L near t=0):}\: g(x)=g(0)+x g'(0) \implies \xi(t,L^{-1})/L = g(0) + tL^{y_t} g'(0)
\end{align*}  
At $t=0$, RHS ind. of $L$. Compute $\xi/L$ for different $L$, $T_c$ found where curves cross. Exponent, $\nu$, gotten by computing $\partial_T(\xi/L)_{(T=T_c)}=L^{y_t}g'(0)/T_c$. Plot log of LHS vs. log of $L$, get straight line with slope $y_t$. 

\subsubsection*{\scriptsize Cumulant expansion}

Use mom. gen. func. 
\begin{align*}
    P_k&=\int dx\, P(x)e^{-ikx} = \sum_{m=0}^\infty \frac{(-ik)^m}{m!}\expval{x^m} \\
    \ln P_k &\equiv \sum_{l=1}^\infty \frac{(-ik)^\ell}{\ell!} \expval{x^\ell}_c
\end{align*}
Equate $P_k$ with the exponentiated $\ln P_k$, compare powers of $(-ik)$. Yields $\expval{x}_c=\expval{x},\:\expval{x^2}_c=\expval{x^2}-\expval{x}^2,...$. $\expval{x^p}$: Draw $p$ dots, connect in all possible ways. Cluster of $m$ dots is $\expval{x^m}_c$, (disjoint: $m=0$). 

Part. func. for interacting gas, $H=\sum_{i=1}^N \frac{\vec{p}_i^2}{2m}+\tilde{u}(\mathbb{Q})$. 
\begin{align*}
    Z = Z_0 \int \frac{d \mathbb{Q}}{V^N} e^{-\beta\tilde{u}} = Z_0 \sum_{m=0}^\infty \frac{(-\beta)^m}{m!} \int \frac{d^3 q_1}{V}\cdots \frac{d^3 q_N}{V}[\tilde{u}(q_1,\dots,q_n)]^m
\end{align*}
so $Z=Z_0 \sum_m (-\beta)^m/m! \expval{\tilde{u}^m}$. For free energy, $\ln Z\sim F$, we get 
\begin{align*}
    \ln Z = \ln Z_0 + \sum_{\ell=1}^\infty \frac{(-\beta)^\ell}{\ell!} \expval{\tilde{u}^\ell}_c
\end{align*}
With assumptions of potential, can now find contr. to $F$ from cumulants. Assuming $\tilde{u}(q...)=\sum_{i<j}u(\vec{q}_i-\vec{q}_j)$ and $u(-q)=u(q)$, the first cum. is 
\begin{align*}
    \expval{\tilde{u}^1}_c = \expval{\tilde{u}}=\frac{N(N-1)}{2}\int d^3 q/V u(\vec{q})
\end{align*}
Diag. for pair-wise int. P pairs for $\expval{\tilde{u}^p}$, connect each by dotted line. Label points, numbers on different pairs can be equal. Merge opints if numbers on different pairs are equal. Find the number of ways, $G$, to assign labels to diagram to get a spec. topology. For a given diagram there is a factor $u(q_i-q_j)$ for each dotted line. An integral $\int d^3 q/V$ for each point. A factor $G$. The sum of $G$ for each order $p$ should equal $(N(N-1)/2)^p$. \textbf{Disconnected and one-particle reducable diagrams don't contribute}. 

\subsubsection*{\scriptsize Cluster expansion}
For hard-core ($u$ big for $q\to0$), the $p$-th term is $\sim \int u^p(q)$. But $p+1$ term is bigger, so the series can't be truncated. Use 
\begin{align*}
    p_1 (-\beta)  + p_2 (-\beta)^2 /2! + \dots + p_p (-\beta)^p /p! = \frac{N(N-1)}{2V} \int d^3 q (e^{-\beta u}-1)
\end{align*}
where the integrand is $f(q)$.


CUMULANT EXPANSION 
CLUSTER EXPANSION 
VIRIAL EXPANSION 


DIFFUSION EQ. RW
MARKOV 


\subsubsection*{\scriptsize Random Walks}

\begin{align*}
    \sum_{R=0}^N P_N(R) &= 1 \\ 
    \expval{R} &= \sum_R R\cdot P_N(R)=NP \\ 
    \expval{R^2} &= \sum_R R^2 P_N(R) = NP(1-p) + N^2 P^2 \\
    \sigma^2 &= P(1-P)N \\ 
    p_N(R) &\approx \frac{1}{\sqrt{2\pi P(1-P)N}}\exp\bracket{-\frac{(R-PN)^2}{2P(1-P)N}}
\end{align*}
so the binomial distribution can be approximated by a Gaussian for sufficiently large $R$ and $N$. 

This gives $\expval{X_N}=Nl(2P-1)$ and $\expval{x_N^2}=4l^2 NP(1-P) + \expval{x_N}^2$ so we get $\expval{x_N^2} - \expval{x_N}^2 = 4l^2 N P(1-P)$.

Now, for $R,N$ large, with $P=1/2$ once again, we can approximate with a Gaussian. Introduce probability density $P_N(x)\equiv P(x)/2l$, since $P(x)$ is the Gaussian distribution for discrete $x$. Then, for $t=N\Delta t$ and $D\equiv \frac{l^2}{2\Delta t}$ we get 

\begin{align*}
    P_N(x) &= \frac{1}{\sqrt{2\pi 2Dt}} e^{-\frac{x^2}{2\cdot 2Dt}}
\end{align*}
and we get a Gaussian distribution with mean $0$ and variance $2Dt$. With $M$ random walkers (particles), all with $x(t=0)=0$, the density of walkers per unit length at time $t$, $\rho(x,t)$, fulfills the Diffusion equation 
\begin{align*}
    \rho(x,t) &= M p(x,t) \\ 
    \pdv{\rho(x,t)}{t} &= D \pdv[2]{\rho(x,t)}{x}
\end{align*}


This result holds for any RW with a symmetric step distribution. Consider general RW with $x(t+\Delta t)=x(t)+l$, where $l$ is now a random length (variable) with a probability distribution $\chi(l)$ that is independent of $t$. The distribution is normalized and symmetric, with $\int dl\,\chi(l)\cdot l^2=a^2$, where the limits are $l=\pm\infty$. For fixed $l$ we previously had $\chi(l)=\frac{1}{2}(\delta(l-a) + \delta(l+a))$. 

Now, what is the distribution $p(x,t+\Delta t)$, given $p(x,t)$. Found by summing all possible paths from $x-l$ up to $x$ between the two times, weighted with the probability of the $l$'s. 
\begin{align*}
    p(x,t+\Delta t) &= \int_{-\infty}^\infty dl\, p(x-l,t)\chi(l) \\ 
    p(x-l,t) &= p(x,t) - l \pdv{p(x,t)}{x} + \frac{l^2}{2!}\pdv[2]{p(x,t)}{x} + \dots \\ 
    \implies p(x,t+\Delta t) &= p(x,t)\int_{-\infty}^\infty dl\,\chi(l) - \pdv{p(x,t)}{x} \cdot 0 + \frac{1}{2} \pdv[2]{p(x,t)}{x} \int_{-\infty}^\infty dl\,\chi(l)\cdot l^2 \\ 
    &= p(x,t) + \frac{a^2}{2} \pdv[2]{p(x,t)}{x}
\end{align*}
Moving $p(x,t)$ to the LHS and dividing by $\Delta t$ yields 
\begin{align*}
    \frac{p(x,t+\Delta t) - p(x,t)}{\Delta t} &= \frac{a^2}{2\Delta t} \pdv[2]{p}{x} \\ 
    \pdv{p}{t} &= D \pdv[2]{p}{x},\quad \text{for}\: \Delta t \to 0 \to
\end{align*}
So all info about microscopics are contained in $D$. 



\subsubsection*{Markov Chains}
Random process where random move only depends on current state of the system, not its history. For a two-state system, let $P_i(n)$ be the probability of finding the system in state $i=1,2$ after $n$ steps. Define the transition probability as $W_{1\gets2}\equiv W_{12}$. For the next step, we can write 
\begin{align*}
    \begin{pmatrix}
        P_1 (n+1) \\
        P_2 (n+1)
    \end{pmatrix}
    &= \begin{pmatrix}
        W_{11} & W_{12} \\ 
        W_{21} & W_{22}
    \end{pmatrix}
    \begin{pmatrix}
        P_1 (n) \\
        P_2 (n)
    \end{pmatrix}
\end{align*}

So $W_{ij}\geq0$, and the total probability of transfering out of a given state must sum to $1$, so $W_{11}+W_{21}=1$. 


The Master's equation, given as 
\begin{align*}
    \dv{P_i}{t} &= \sum_{j} \closed{\omega_{ij}P_j - \omega_{ji}P_i},\quad \omega_{ii}=0
\end{align*}

\end{multicols*}
\end{document}