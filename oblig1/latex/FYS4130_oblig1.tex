\documentclass[reprint,english,notitlepage,aps,nobalancelastpage,nofootinbib]{revtex4-1}
\usepackage[utf8]{inputenc}
\usepackage[english]{babel}
\usepackage{physics,amssymb}
\usepackage{graphicx}
\usepackage{xcolor}
\usepackage{hyperref}
\usepackage{tikz}
\usepackage{listings}
\usepackage{subfigure}
\usepackage{amsmath,mathtools}
\usepackage{amsbsy}
\usepackage{enumitem}
\usepackage{bbold}

\graphicspath{{../plots/}}

\hypersetup{
    colorlinks,
    linkcolor={red!50!black},
    citecolor={blue!50!black},
    urlcolor={blue!80!black}}

\lstset{
	inputpath=,
	backgroundcolor=\color{white!88!black},
	basicstyle={\ttfamily\scriptsize},
	commentstyle=\color{magenta},
	language=Python,
	morekeywords={True,False},
	tabsize=4,
	stringstyle=\color{green!55!black},
	frame=single,
	keywordstyle=\color{blue},
	showstringspaces=false,
	columns=fullflexible,
	keepspaces=true}


\newcommand{\closed}[1]{\left(#1\right)}
\newcommand{\bracket}[1]{\left[#1\right]}

\newcommand{\kt}{\kappa_T}
\renewcommand{\cp}{c_P}
\newcommand{\cv}{c_V}
\renewcommand{\a}{\alpha}

\newcommand{\tmdv}[4]{\closed{\pdv{#1}{#2}}_{#3,#4}}
\newcommand{\jacobian}[2]{\pdv{(#1)}{(#2)}}
\renewcommand{\d}{\mathrm{d}}

\begin{document}
\begin{center}
\title{\Huge FYS4130 - Oblig 1}
\author{\large Vetle A. Vikenes}
\date{\today}
\noaffiliation


\maketitle
\end{center}
\onecolumngrid
\centering
\section*{\large Task 1 - Black box numerical method}
We want to find the isothermal compressibility at constant $T$ and $N$,
\begin{align} \label{eq:kappa black box}
	\kt = -\frac{1}{V}\closed{\pdv{V}{P}}_{T,N},
\end{align}
with a black box numerical method which gives us values for $P$ and $N$ as function of intput parameters $T,\,V,$ and $\mu$ which we may vary. The derivatives we're able to compute with our numerical method is the partial derivative of either $P$ or $N$ with respect to one of the input parameters, when the other two are held constant. We must therefore rewrite equation \eqref{eq:kappa black box}, such that it only contains such derivatives. To begin, we use the chain rule on the reciprocal of the partial derivative in equation \eqref{eq:kappa black box}
\begin{align*}
	\tmdv{P}{V}{T}{N} &= \jacobian{P,T,N}{V,T,N}=\jacobian{P,T,N}{V,T,\mu}\cdot\jacobian{V,T,\mu}{V,T,N} \\
	&= \jacobian{P,N,T}{V,\mu,T}\cdot\jacobian{\mu,V,T}{N,V,T}=
	\jacobian{P,N,T}{V,\mu,T} \Big/ \tmdv{N}{\mu}{V}{T}.
\end{align*}
The last factor is a partial derivative we're able to compute, by computing the change in $N$ as we vary $\mu$ only. For the other factor, we expand the jacobian, using its definition 

\begin{align}
	\jacobian{P,N,T}{V,\mu,T} &= \tmdv{P}{V}{\mu}{T}\tmdv{N}{\mu}{V}{T}-\tmdv{P}{\mu}{V}{T}\tmdv{N}{V}{\mu}{T}. \label{eq:d PNT_d VmuT}
\end{align}
We see that the four partial derivatives in equation \eqref{eq:d PNT_d VmuT} are taken with respect to either $V$, with the other input parameters held constant, or $\mu$ with the other input parameters held constant. Measuring the change in $P$ and $N$ with these constraints is thus something our numerical method is capable of, and we have successfully reduced $\kt$ into multiple partial derivatives we're able to solve. The resulting expression for $\kt$ becomes  

\begin{align*}
	\kt &= -\frac{1}{V}\bracket{\jacobian{P,N,T}{V,\mu,T} \Big/ \tmdv{N}{\mu}{V}{T}}^{-1}=-\frac{1}{V} \tmdv{N}{\mu}{V}{T} \Big/
	\jacobian{P,N,T}{V,\mu,T} \\
	&= -\frac{1}{V} \tmdv{N}{\mu}{V}{T} \bracket{\tmdv{P}{V}{\mu}{T}\tmdv{N}{\mu}{V}{T}-\tmdv{P}{\mu}{V}{T}\tmdv{N}{V}{\mu}{T}}^{-1}.
\end{align*}

\section*{\large Task 2 - Partial derivative}
We want to rewrite the partial derivative 
\begin{align*}
	\tmdv{P}{U}{G}{N}.
\end{align*}
In this task we will need the standard set of second derivatives given in Swendsen, and we list them here for convenience
\begin{align}
	\a &= \frac{1}{V} \tmdv{V}{T}{P}{N} \label{eq:alpha} \\ 
	\kt &= -\frac{1}{V} \tmdv{V}{P}{T}{N} \label{eq:kappa} \\ 
	\cp &= \frac{T}{N} \tmdv{S}{T}{P}{N} \label{eq:cp} \\ 
\end{align}
We begin by applying the chain rule to the Jacobian, which enables us to get a partial derivative containing $U$ and $G$ in the denominator only. 

\begin{align} 
	\tmdv{P}{U}{G}{N} &= \jacobian{P,G,N}{U,G,N} = \jacobian{P,G,N}{P,T,N}\cdot\jacobian{P,T,N}{U,G,N}= \tmdv{G}{T}{P}{N} \jacobian{P,T,N}{U,G,N} \nonumber \\
	&= -S \Big/\jacobian{U,G,N}{P,T,N}, \label{eq:pdv first expansion}
\end{align}
where we used the differential form of the Gibbs free energy, $\mathrm{d}G=-S\mathrm{d}T+V\mathrm{d}P+\mu\mathrm{d}N$, to solve the first partial derivative of $G$ with respect to $T$ at constant $P$ and $N$. The second factor was written in terms of the reciprocal such that the thermodynamic potentials in the Jacobian appear in the nominator. 

We will now find an expression for $\pdv*{(U,G,N)}{(P,T,N)}$. To avoid partial derivatives containing $U$ and $G$ simultaneously, we expand the expression by using the definition of Jacobians. 

\begin{align}
	\jacobian{U,G,N}{P,T,N} &= \tmdv{U}{P}{T}{N} \tmdv{G}{T}{P}{N} - \tmdv{U}{T}{P}{N}\tmdv{G}{P}{T}{N} \nonumber \\ 
	&= -S\tmdv{U}{P}{T}{N} - V \tmdv{U}{T}{P}{N}, \label{eq:d UGN_d PTN}
\end{align}
where the last equation follow from the definition of $\d G$ and the partial derivatives of it. 

For the partial derivative of $U$ with respect to $T$ at constant $P$ and $N$, we consider the differential form of fundamental relation in the energy representation, which at constant $N$ becomes 

\begin{align}
	\d U &= T \d S - P \d V \nonumber \\ 
	\implies \tmdv{U}{T}{P}{N} &= T \tmdv{S}{T}{P}{N} - P \tmdv{V}{T}{P}{N} \nonumber \\ 
	&= N\cp - PV\a, \label{eq:dU_dT}
\end{align}
where we used equations \eqref{eq:cp} and \eqref{eq:alpha} for the two partial derivatives. 

For the partial derivative of $U$ with respect to $P$ with $T$ and $N$ held constant we apply the chain rule
\begin{align}
	\tmdv{U}{P}{T}{N} &= \jacobian{U,T,N}{P,T,N} = \jacobian{U,T,N}{V,T,N}\cdot \jacobian{V,T,N}{P,T,N} = \tmdv{U}{V}{T}{N} \tmdv{V}{P}{T}{N} \nonumber \\
	&= \tmdv{U}{V}{T}{N}(-V\kt). \label{eq:dU_dP}
\end{align}
Equation \eqref{eq:kappa} was used to rewrite the second partial derivative. For the other partial derivative, we once again use the previouslly mentioned expression for $\d U$ with $N$ held constant 
\begin{align} \label{eq:dU_dV}
	\tmdv{U}{V}{T}{N} &= T \tmdv{S}{V}{T}{N} - P.
\end{align}
To proceed with the final partial derivative we will first use a Maxwell relation. We notice that $S$ is differentiated with respect to $V$ at constant $T$ and $N$, so we can use the differential form of the Helmholtz free energy to derive the Maxwell relation. 
\begin{align*}
	\d F &= -S \d T -P \d V + \mu \d N \implies -\tmdv{F}{T}{V}{N} = S \\ 
	\tmdv{S}{V}{T}{N} &=-\bracket{\pdv{V}\tmdv{F}{T}{V}{N}}_{T,N} = -\bracket{\pdv{T}\tmdv{F}{V}{T}{N}}_{V,N} = \tmdv{P}{T}{V}{N}.
\end{align*} 
We rewrite the last partial derivative using the chain rule 
\begin{align}
	\tmdv{P}{T}{V}{N} &= \jacobian{P,V,N}{T,V,N} = \jacobian{P,V,N}{P,T,N}\cdot \jacobian{P,T,N}{T,V,N} = -\jacobian{V,P,N}{T,P,N}\cdot \jacobian{P,T,N}{V,T,N} \nonumber \\ 
	&= -\tmdv{V}{T}{P}{N} \Big/ \tmdv{V}{P}{T}{N} = -V\a \big/ (-V\kt) = \frac{\a}{\kt}, \label{eq:alpha_kappa}
\end{align} 
where in the last step equations \eqref{eq:alpha} and \eqref{eq:kappa} were used to rewrite the nominator and denominator, repsectively. Equation \eqref{eq:dU_dP} can now be solved, by inserting equation \eqref{eq:alpha_kappa} into equation \eqref{eq:dU_dV} 
\begin{align*}
	\tmdv{U}{P}{T}{N} &= -V\kt \tmdv{U}{V}{T}{N} = -V\kt \closed{T\frac{\a}{\kt} - P} \\ 
	&= -VT \a + PV\kt 
\end{align*}

Inserting equation \eqref{eq:dU_dP} and \eqref{eq:dU_dT} into equation \eqref{eq:d UGN_d PTN} we get 
\begin{align}
	\jacobian{U,G,N}{P,T,N} = -S \closed{-VT \a + PV\kt} - V \closed{N\cp - PV\a} = -V\bracket{SP\kt + N\cp -ST\a - PV\a} \label{eq:d UGN_d PTN final expression}
\end{align}

Finally, inserting equation \eqref{eq:d UGN_d PTN final expression} into equation \eqref{eq:pdv first expansion} we arrive at the final expression 
\begin{align}
	\tmdv{P}{U}{G}{N} &= -S \bracket{\jacobian{U,G,N}{P,T,N}}^{-1} = \frac{S/V}{SP\kt + N\cp -ST\a - PV\a}
\end{align}


\section*{\large Task 3}
 


\subsection*{a)}





\end{document}
